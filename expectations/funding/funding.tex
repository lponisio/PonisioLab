\documentclass[12pt]{article}
\usepackage[top=0.8in, bottom=0.8in, right=0.8in, left=0.8in,
paperwidth=8.5in, paperheight=11in, nohead]{geometry}
\geometry{letterpaper}
\usepackage[pdftex]{graphicx}
\usepackage{color}
\usepackage[normalem]{ulem}
\usepackage{amssymb}
\usepackage{amsmath}
\usepackage{epstopdf}
\usepackage{setspace}
\usepackage{mdwlist}
\usepackage{hyperref}

\title{Funding}
\author{Lauren C. Ponisio}

\begin{document}
\maketitle

Acquiring funding to support research and salaries can the be most
daunting, stress-inducing aspect of doing science, especially given
the current political climate.

\section{General advice}

\begin{enumerate}
\item Start looking for opportunities early (check team drive for a
  list of opportunities for graduate students and post-docs).
\item Keep your project expenses trackable. You will probably be able
  to raise $\$5000-10000$ a year as a graduate student
  ($\$10000-50000$ as a post doc), if your project is unfeasible under
  that budget, reassess
\item Early on, take every opportunity you can to read successful
  proposals (i.e., those that were funded) written by others. Ask
  yourself, what makes this a good proposal? Do the same with
  proposals that were rejected. Ask yourself, just what is it about
  this proposal that kept it from being funded. 
\item \textbf{Work together}. Ask others to read your grants and offer
  to read theirs. Be transparent about what you are applying
  for. Share successful grants. Though sometimes it feels like you are
  competing for grants since many lab members are applying for the
  same grants, remember we are all in the same boat. We are all
  working together to conserve biodiversity and expand our knowledge
  of the natural world. Any funding coming into the lab helps you
  directly or indirectly. As lab PI I would do my best to equalize
  funding if a few students struggle in acquiring grants
\item As PI, I will give no preference to any student/post-doc's
  projects. All common lab funding will be allocated equally between
  students based on current lab finances. \textbf{To receive lab
    funding, students must submit a short proposal.}
\item If, as mentioned above, some students are more successful than
  others at acquiring funding, we will discuss diverting funds to the
  students in need of funding together, with maximal transparency.
\end{enumerate}

\section{How to write a fund-able proposal}

\begin{enumerate}
\item know your audience! \textbf{Address the funding call!}
\item Find big gaps in knowledge (transformative science) (make sure
  it is mostly doable)
\item Tell a good story
\item be controversial, shocking and tenacious, but not too far
\item preliminary data is good, but don't give too much of a story
\item have to have the right structure for the funding call
\item \textbf{unfundable proposals} missed the mark; did not address
  the funding call; poorly written
\item Proposal ``formula''
  \begin{itemize}
  \item hypothesis, bolded and justified
  \item methods can be cited instead of explained
  \item carefully chosen display items
  \item section on how results will be interpreted
  \item results of previous grants (like for NSF) and be at the end it
    not super related
  \end{itemize}
\end{enumerate}

\section{Salary}
Salaries are based on a step system (for post-docs, this
is based on years post phd). You all do equally amazing work and
deserve to have the reflected in your equal salaries. 

\textbf{Try to bring in fellowships. They will demonstrate you can
  fund yourself, build your CV and raise your salary}

I have a collection of successful grants. Ask me to see specific
grants.

\section{Travel}
\begin{enumerate}
\item Students and post-docs must attempt to self-fund travel to
  conferences etc. Funds can be applied for from the conference
  itself, as well as UO (for graduate students). 
\item Two conferences a year, at most, are recommended
  for senior graduate students and postdocs
\item To receive lab funds, students and postdocs are require to
  submit a brief budget and proposal describing why this conference is
  important to their careers. Per-Diem will not be covered since
  ecology grants are not large enough to accomodate more than 1k per
  conference, nor more than two conferences a year. If students and
  postdocs must be presenting lab-related work and must have tried to
  fund raise and failed to qualify for lab funding if it is not
  explicility budged on a grant.
\end{enumerate}

\end{document}

%%% Local Variables:
%%% mode: latex
%%% TeX-PDF-mode: t
%%% End:


