\documentclass[12pt]{article}
\usepackage[top=0.8in, bottom=0.8in, right=0.8in, left=0.8in,
paperwidth=8.5in, paperheight=11in, nohead]{geometry}
\geometry{letterpaper}
\usepackage[pdftex]{graphicx}
\usepackage{color}
\usepackage[normalem]{ulem}
\usepackage{amssymb}
\usepackage{amsmath}
\usepackage{epstopdf}
\usepackage{setspace}
\usepackage{mdwlist}
\usepackage{hyperref}
\usepackage{xr}
\externaldocument{../general_advice/generalAdvice}
\externaldocument{../funding/funding}

\title{Ponisio lab personnel policy}
\author{Lauren C. Ponisio}

\begin{document}

\maketitle

\section{PI}
\begin{enumerate}
\item Forward the professional interests of the members of the lab.
\item Foster and encourage an atmosphere of inclusivity, exploration,
  learning, and good communication.
\item Ultimately responsible for research funding, accounting,
  scientific oversight, and training and oversight of personnel
\item Be available to talk about and comment on student and post-doc
  research ideas and grant applications.
\item Be available to discuss personnel and scientific issues
\item Answer to the funding agencies. Ensure that public dollars are
  well-spent; that the science that emerges from my lab is productive,
  rigorous, and cutting edge.
\item I must approve all abstracts, manuscripts, or any representation
  of the research that comes from the lab, before it leaves the door
  for approval by collaborators or submission to meetings or journals.
\item Be timely, conscientious, and constructive in my comments on
  abstracts and manuscripts.
\item Be timely in writing letters of recommendation.
\end{enumerate}

\section{Undergraduates}
\begin{enumerate}
\item Actively participate in laboratory group functions including lab
  meeting where possible.
\item work as a group to support a sustainable schedule and adhere to
  it. A minimum of 3 hours per week for 1 unit is required. 
\item if scheduling conflicts arise, be proactive, and communicative.
\item laboratory work will initially involve duties that support the
  work of lab technicians, students, and post-docs. Self-sufficient
  projects are the goal, but may take several quarters to be
  realized. Assignment of an individual project is possible only after
  several quarters of dedicated work in supporting roles, sufficient
  demonstration of commitment, enthusiasm and attention to
  detail. Opportunities may arise to be included as an author on
  publications that emerge from lab contributions when appropriate.
\item If/when opportunities arise to attend research conferences,
  students should seek out funding to attend such conferences (from
  UO, the conference, etc.), see
  \href{https://urds.uoregon.edu/cure-awards/conference}{Conference
    Travel Awards}.
\end{enumerate}

\section{Lab Technicians}
\begin{enumerate}
\item Is expected to be a leader in the laboratory, helping to foster
  an atmosphere of inclusively, exploration, and learning
\item Enforcing lab safety, sertility and PPE. Organize annual safety audit. 
\item Keep equiment maintained (pipettes, biosaftey cabinet etc.). 
\item Manage and inventory chemicals. 
\item Faithfully commit to the contracted work hours but set your own schedule.
\item Be flexible and available and willing to contribute beyond
  scheduled hours in emergencies or crunch times; hours will be
  compensated during non-critical periods.
\item Be proactive in troubleshooting and technology development
\item Contribute to training new lab personnel
\item May take opportunities to work on self-inspired or self-motivated
  projects, after assigned duties are fulfilled.
\item Follow priorities set by PI or lab, keep track of hours spent on
  tasks if asked to do so (often for future grant planning).
\item Inform the PI when planning to take vacation days two weeks in
  advance for multi-day vacations, a week in advance for a single
  vacation day (i.e., a Friday).
\item Health days are for when you (or a close relative, partner,
  child etc.) are unwell, or need an extended doctor's
  visit. Call/email the PI the day you wake up sick and inform them
  that you are not well. For regular doctors visits up to four hours
  can be taken. Health days cannot be taken with vacation days unless
  you become unwell before a planned vacation.
\item Oregon offers 12 weeks paid health-realted leave that can be
  taken annually (contact HR for details, and
  \href{https://www.oregon.gov/boli/workers/pages/oregon-family-leave.aspx}{Oregon
    Family Leave Act}).
\end{enumerate}


\section{Graduate Students}
\begin{enumerate}
\item Is expected to be a leader in the laboratory, in helping foster
  an atmosphere of inclusivity, exploration, and  learning
\item Be responsible for own schedule and goal setting (see General Advice
  Section \ref{sec:goals})
\item Seek out additional mentors and collaborators (see General
  Advice Section \ref{sec:mentors}), and ask for help when needed (see
  General Advice Section \ref{sec:help})
\item develop your research ideas and learn to talk about them
  effectively (see General Advice Section \ref{sec:research}-
  \ref{sec:talkScience})
 \item Participate yearly in educational outreach related to the lab's mission.
\item Be flexible and available and willing to work beyond usual
  schedule in emergencies or crunch times (i.e., field seasons).
 \item During the field season, prioritize fieldwork above other
   commitments (we have a narrow window to collect data each
   year). Take vacation after all data has been collected unless a
   substitute can be arranged. Because pollinator work is weather
   dependent, work may shift to weekends (and days off to week days)
   so avoid weekend commitments that cannot be shifted during the peak
   field season.
\item Mentor undergrads in the lab/field. PhD students should plan to
  mentor one undergraduate thesis before they graduate.
\item contribute to assigned tasks (keeping website up-to-date,
  tidying up lab, etc.)
\item Expected to write fellowship grants throughout their tenure for
  research money, stipend funds, travel funds, etc. (See Funding)
\item Attend research conferences, especially in the last few years of
  their tenure. Lab funds for travel are limited, so you must try to
  fund yourself through grants from the conference and department.
\item Be generally present in the lab during the academic school
  year. For students who prefer to work-from-home, come to campus 1-2
  days a week for lab meeting and any in-person meetings. 
\item PhD students should expect to finish three chapters before
  graduating, masters students 1. 
\item Submit their chapter(s) for publication before conferring their
  degree (potential exceptions for masters degrees if using bee
  community data). If the person is likely to move on and be unable to
  finish a publication, they must nominate a ``second in command'' for
  the publication before graduating to agree to take the study over
  the publication finish line including submission and revision.
\end{enumerate}

\section{Post-docs}
\begin{enumerate}
\item Is expected to be a leader in the laboratory, in helping foster
  an atmosphere of inclusivity, exploration and learning
 item Lead lab projects. Aim to publish 1 paper a year.
\item  Responsible for own schedule, goal setting (see General
  Advice Section \ref{sec:goals}), and to work their contracted hours
\item Be flexible and available and willing to work beyond personal
  schedule in emergencies or crunch times
\item Seek out additional mentors and collaborators (see General
  Advice Section \ref{sec:mentors}), and ask for help when needed (see
  General Advice Section \ref{sec:help})
\item Develop your research ideas and learn to talk about them
  effectively (see General Advice Section \ref{sec:research}-
  \ref{sec:talkScience})
\item  Write fellowship grants throughout their tenure
  for research money, stipend funds, travel funds, etc... (See Funding)
\item is typically hired to work on a specific project, on which
  their goal should be to earn primary authorship of emergent
  publications.
\item Mentor undergrads and graduate students
\item contribute to assigned tasks (keeping website up-to-date,
  tidying up etc.)
\item Side projects are possible and encouraged as long as either 1)
  the project that they were hired to contribute to is complete or
  rapidly progressing, and 2) the postdoc secures their own research
  funding.
\item Mentor graduate students and/or undergraduate trainees.
\item Encouraged to attend research conferences. Lab will only pay
  for travel and fees if funds are available, and if the postdoc is
  primary author and presenting (platform talk or poster). Even if
  this is the case, postdoc is expected to apply for travel funds if
  there are some available (i.e., through UO, the conference etc.).
\item Be generally present in the lab during the academic school year,
  unless previously arranged with the PI.
\item If on a specific grant, expected to spent 90\% of more of
  their contracted hours on lab-related projects (at most 10\%
  on phd work, other collaborations etc.). If on a fellowship,
  this split is more liberal, though the postdoc must still
  spend the majority of their time on lab-related projects.
\item Health days are for when you (or a close relative, partner,
  child etc.) are unwell, or need an extended doctor's visit. Remeber
  to fill out your timesheet with any health days taken. Oregon offers
  12 weeks paid Health-realted leave that can be taken annually
  (contact HR for details, and
  \href{https://www.oregon.gov/boli/workers/pages/oregon-family-leave.aspx}{Oregon
  Family Leave Act}).
\end{enumerate}

\end{document}

%%% Local Variables:
%%% mode: latex
%%% TeX-PDF-mode: t
%%% End:


