\documentclass[12pt]{article}
\usepackage[top=0.8in, bottom=0.8in, right=0.8in, left=0.8in,
paperwidth=8.5in, paperheight=11in, nohead]{geometry}
\geometry{letterpaper}
\usepackage[pdftex]{graphicx}
\usepackage{color}
\usepackage[normalem]{ulem}
\usepackage{amssymb}
\usepackage{amsmath}
\usepackage{epstopdf}
\usepackage{setspace}
\usepackage{mdwlist}
\usepackage{hyperref}

\title{Funding}
\author{Lauren C. Ponisio}

\begin{document}
\maketitle

Acquiring funding to support research and salaries can the be most
daunting, stress-inducing aspect of doing science, especially given
the current political climate.

\section{General advice}

\begin{enumerate}
\item Start looking for opportunities early (see 
  \href{https://docs.google.com/spreadsheets/d/1X2Dyz4HW13hsTmNoUvSoMYOmyobVVtfoGtEtdS0LQtI/edit?usp=sharing}{funding
    spreadsheet}) 
\item Keep your project expenses trackable. You will probably be able
  to raise $\$5000-10000$ a year as a graduate student
  ($\$10000-50000$ as a post doc), if your project is unfeasible under
  that budget, reassess
\item Early on, take every opportunity you can to read successful
  proposals (i.e., those that were funded) written by others. Ask
  yourself, what makes this a good proposal? Do the same with
  proposals that were rejected. Ask yourself, just what is it about
  this proposal that kept it from being funded. (see lab grant
  repository for examples)
\item \textbf{Work together}. Ask others to read your grants and offer
  to read theirs. Be transparent about what you are applying
  for. Share successful grants. Though sometimes it feels like you are
  competing for grants since many lab members are applying for the
  same grants, remember we are all in the same boat. We are all
  working together to conserve biodiversity and expand our knowledge
  of the natural world. Any funding coming into the lab helps you
  directly or indirectly. As lab PI I would do my best to equalize
  funding if a few students struggle in acquiring grants
\item As PI, I will give no preference to any student/post-doc's
  projects. All common lab funding will be allocated equally between
  students based on current lab finances. If, as mentioned above, some
  students are more successful than others at acquiring funding, we will
  discuss diverting funds to the students in need of funding
  together, with maximal transparency. 
\end{enumerate}

\section{How to write a fund-able proposal}

\begin{enumerate}
\item know your audience! \textbf{Address the funding call!}
\item Find big gaps in knowledge (transformative science) (make sure
  it is mostly doable)
\item Tell a good story
\item be controversial, shocking and tenacious, but not too far
\item preliminary data is good, but don't give too much of a story
\item have to have the right structure for the funding call
\item \textbf{unfundable proposals} missed the mark; did not address
  the funding call; poorly written
\item Proposal ``formula''
  \begin{itemize}
  \item hypothesis, bolded and justified
  \item methods can be cited instead of explained
  \item carefully chosen display items
  \item section on how results will be interpreted
  \item results of previous grants (like for NSF) and be at the end it
    not super related
  \end{itemize}
\end{enumerate}

\section{Salary}
All graduate students on GSRs and post-docs in the lab will be funded
at the same rate (and with post-docs, a increase in salary set by the
UC each year). You all do equally amazing work and deserve to have the
reflected in your equal salaries. If someone brings in a fellowship
that pays at a higher rate, I will do my best to raise other salaries
to that rate, depending on available lab funding.

\textbf{Try to bring in fellowships. They will demonstrate you can
  fund yourself, build your CV and raise your salary}

(see \href{https://docs.google.com/spreadsheets/d/1X2Dyz4HW13hsTmNoUvSoMYOmyobVVtfoGtEtdS0LQtI/edit?usp=sharing}{funding
  spreadsheet})

If you are constantly worrying about finances (as I was during
graduate school in the Bay Area), this will affect your ability to do
science! If the lab feels their salaries are not enough to cover basic
expenses for living in the area, we can come together as a lab and
discuss a new rate for everyone. We are all a research team, so it is
important to remember I am balancing salaries with research
funds. Money spent on lab salaries may take away from funds that could
be used to buy research supplies and fund field assistants. Since the
success of the lab's research benefits all, we all need to keep that
in mind when we discuss funding allocations.

\section{Safety net fund}
We have all experienced the shock of having a sudden financial expense
from things like a medical procedure, or needing to take care of a
family member. Some of us are lucky enough to have a family or
significant other or previous savings to help with that expense,
others are not. Though the specifics of how this fund will work are
not entirely worked out, if you have such an expense and do not have
an external safety net, some talk to me and we will see if we can
supplement your salary (for example though a summer GSR). It seems one
of the largest impediments for students to succeed in graduate school
is that they do not have outside financial support. The goal of this
fund is to help equalize that support.

\end{document}

%%% Local Variables:
%%% mode: latex
%%% TeX-PDF-mode: t
%%% End:


