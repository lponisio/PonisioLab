\documentclass[12pt]{article}
\usepackage[top=0.8in, bottom=0.8in, right=0.8in, left=0.8in,
paperwidth=8.5in, paperheight=11in, nohead]{geometry}
\geometry{letterpaper}
\usepackage[pdftex]{graphicx}
\usepackage{color}
\usepackage[normalem]{ulem}
\usepackage{amssymb}
\usepackage{amsmath}
\usepackage{epstopdf}
\usepackage{setspace}
\usepackage{mdwlist}
\usepackage{url}
\usepackage{xr}
\externaldocument{generalAdvise}
\externaldocument{funding}

\title{Resources at UCR}
\author{Lauren C. Ponisio}

\begin{document}

\maketitle

\section{Postdoctoral, Graduate and Undergraduate  Training Resources}

\begin{enumerate}
\item Success and Leadership Skills for Academe (SALSA) offers five
  workshops to enhance the success of new and current junior faculty,
  professional researchers, and postdoctoral fellows by providing
  guidance for the challenges in grantsmanship, publications, and
  laboratory management.  Provides professional development and
  mentoring for all UCR research faculty and academic personnel.

\item UCR Career Center offers comprehensive resources, programs, and
  counseling on career development and employment to help undergrads,
  graduate students, postdocs, and alumni make informed decisions
  about their futures. The Center is open to all trainees
  campus-wide. In addition, the Center organizes career fairs,
  information sessions, and workshops.  

\item GradSuccess provides a variety
  of services to meet the needs of UCR’s diverse graduate student
  population. Housed in Graduate Division, GradSuccess offers
  programs, workshops, seminars, and consultations. GradSuccess
  supports graduate students at every stage of their study and is
  concerned with helping students become successful professionals.

\item GradQuant offers training in probability and statistical
  inference, statistical software and computing, math for statistics,
  data management, and professional ethics in the management and
  analysis of data. The educational support ranges from remedial and
  introductory methods to advanced, specialized training. Graduate
  students can receive quantitative support via one-on-one
  consultations with GradQuant staff and via workshops offered by
  staff and outside statistical experts.

\item Software Carpentry is a non-profit organization dedicated to
  “democratizing data science” by offering workshops taught by
  volunteer instructors. \textbf{All are encouraged to get a
    certification, the lab will cover the cost}. A Software Carpentry
  workshop is hands-on two-day event that covers the core skills
  needed to be productive in a small research team. Short tutorials
  alternate with practical exercises, and all instruction is done via
  live coding.

\item Graduate Writing Center offers writing support and instruction
  to all UCR graduate students and postdoctoral scholars through free
  workshops and writing consultations. They provide writing assistance
  in any academic genre during any stage of the writing process
  (e.g. abstracts, grant applications, developing journal articles, CV
  and resume basics, and many more).

\item Teaching Training:  The Teaching Assistant Development Program
  (TADP) trains all TAs at UCR, organizing a major orientation and
  offering workshops throughout the year. TAs are offered one-to-one
  mentorship and the opportunity to observe peers whose teaching has
  been recognized as outstanding.

\end{enumerate}


\section{Undergraduate research}
\begin{enumerate}
\item Chancellor’s Research Fellows Program is designed to enhance and
  encourage the development of faculty-mentored research and creative
  activity among our undergraduate student population. The fellowship
  provides up to \$5,000 to support research expenses, travel to
  present at conferences, and a
  stipend. \url{http://ssp.ucr.edu/chancellor_fellowship/}
\item Student Minigrants fund up to \$1,000 for research expenses,
  including travel to present at national research conferences.  The
  deadline for spring funding is in February each year .  Workshops on
  how to write a proposal for a Minigrant will be held in January in
  year. \url{http://ssp.ucr.edu/student_grant_opportunities/}
\item The Undergraduate Research Journal provides an opportunity for
  students to publish their research findings.
\item The Undergraduate Research Symposium highlights faculty mentored
  undergraduate research.  Presentation Workshops will be hosted
  throughout winter quarter to assist students in preparing for this
  campus event to be held in May each year \url{http://ssp.ucr.edu/symposium/}
\item Academic Internships at UC Riverside provides a variety of
  opportunities for eligible students to earn course credit for
  substantive internships associated with their academic and career
  goals.  Internship opportunities are available in Washington,
  D.C. (UCDC program), Sacramento, CA (UC Center in Sacramento), and
  in the local Inland Empire region (faculty-led internships offered
  by academic
  departments). \url{http://ssp.ucr.edu/academic_internships.html}
\item The Global Issues Forum provides a space for students and
  faculty from across all disciplines to work together on common goals
  of internationalizing the curriculum and integrating international
  experiences with academic experiences.  In its inaugural year
  (15-16), the Global Issues Forum hosted events highlighting
  immigration, the European economic crisis, gender equity, drug
  trafficking, food security, and more.  The events include scholars
  and practitioners from on and off campus, leading in-depth
  discussions on current global
  issues. \url{http://ucrgif.wix.com/globalissuesforum}
\item Leadership Pathway:A two-year program for selected students to
  develop their understanding of leadership and their leadership
  skills through participation in at least two of three courses taught
  by UCR faculty members Ronald O. Loveridge, Thomas Sy, and Elaine
  Wong. The program is based on the “3 E’s” of leadership development:
  Education, Experience, and
  Exemplars. \url{http://leadershippathway.ucr.edu/ }
\item National Prestigious Scholarships and Awards
  Program:Undergraduate Education provides support and guidance to
  prepare qualified students to be competitive in applying for
  national scholarships, fellowships, and internships. This website
  provides a database of national award opportunities. Students are
  encouraged to identify awards early in their university experience
  so that they may focus on preparing a competitive
  profile. \url{http://ssp.ucr.edu/scholarships/ }
\item R'courses: R’Courses are 1 unit, S/NC offerings facilitated by
  UCR undergraduate students. Each course has a faculty instructor of
  record who provides mentoring and support behind the scenes. These
  courses are an opportunity to develop leadership skills, innovate
  the undergraduate curriculum, and promote democratic, experiential
  education on campus. Interested students submit a proposal 2
  quarters before they offer the course, and complete a facilitator
  training course 1 quarter before. \url{http://rcourses.ucr.edu}
\item Service learning: UCR joins research universities across the
  country in affirming its commitment to community-based research and
  service-learning.  The 2020 Strategic Plan aspires that “every
  undergraduate student should have the opportunity to be involved in
  a community engagement activity (e.g., service learning course,
  community engagement research project, structured volunteerism,
  internship, education abroad) that has a true academic component.”
  The Office of Undergraduate Education is launching a new initiative
  to scale up academic community engagement by providing logistical
  and service-learning placement support for departments and faculty
  wishing to embed a service-learning component into existing
  courses. \url{servicelearning@ucr.edu}
\item Academic Internships: Academic Internships at UC Riverside
  provides a variety of opportunities for eligible students to earn
  course credit for substantive internships associated with their
  academic and career goals.  Internship opportunities are available
  in Washington, D.C. (UCDC program), Sacramento, CA (UC Center in
  Sacramento), and in the local Inland Empire region (faculty-led
  internships offered by academic
  departments). \url{http://ssp.ucr.edu/academic_internships.html}


\end{enumerate}

\section{Data Publication and Archiving Resources}
\begin{enumerate}
\item California Digital Library. The California Digital Library (CDL)
  provides the infrastructure and support for publishing and accessing
  digital information in the ten-campus University of California
  System. One of the major services provided by CDL, eScholarship,
  serves as an institutional repository and publishing platform for UC
  scholars. Our project will make publications accessible to the
  general public via eScholarship.

\item University of California Curation Center. The University of
  California Curation Center (UC3) works as a chapter of the
  California Digital Library to broker access to tools that ensure
  long-term viability and usability of curated digital content for the
  ten UC campuses. Two relevant tools include: DASH, a self-service
  data curation tool; and Merritt, the UC System's digital
  preservation repository for the management, archiving, and sharing
  of digital content by the UC community. Merritt allows access to the
  general public via persistent URLs, provides tools for long-term
  data management, and permits permanent storage. The service has
  built-in contingencies for disaster recovery, including redundancy
  and recovery plans. Merritt is built upon a Micro-Services approach
  to digital curation: the curation function is devolved into a set of
  independent, but interoperable, services that embody curation values
  and strategies. Since each of the services is small and
  self-contained, they are collectively easier to develop, deploy,
  maintain, and enhance.
\end{enumerate}

\section{Resources for Promoting Diversity and Inclusivity}

\begin{enumerate}
\item Leading Through Diversity Partnership for Faculty Equity and
  Diversity was funded by an NSF Partnership for Adaptation,
  Implementation, and Dissemination Award (PAID) which supports the
  analysis, adaptation, dissemination and use of existing innovative
  materials and practices that have been demonstrated to be effective
  in increasing representation and participation of women in academic
  science and engineering careers.

\item UC Family Friendly Policies: Family accommodation is fundamental
  to an equitable and productive academic environment. UC has
  established policies and programs, codified in the Creating a Family
  Friendly Department: Chairs and Deans Toolkit of essential practical
  information for department chairs and deans, to assist faculty and
  other academic appointees in balancing the demands of work and
  family.

\item Professional Societies for Women UCR faculty participate in a
  number of organizations that promote and encourage women in STEM
  fields, support professional growth, mentoring and equity.  UCR
  established the Inland Empire chapter of the Association for Women
  in Science, has a UCR chapter of the Society of Women Engineers, and
  faculty participate in specialized Associations for Women in
  Computing, Mathematics, Neuroscience, Engineering, the American
  Medical Women’s Association, and the Committee on the Status of
  Women in Astronomy.

\item University of California Institute for Mexico and the United
  States (UC MEXUS) located at UCR is an institute designed to
  increase the quantity, visibility, and effectiveness of
  Mexico-United States projects in the University; to strengthen and
  develop research, exchange programs, and teaching; to support and
  coordinate interdisciplinary and inter-campus projects; to encourage
  and enable collaborative approaches by UC and Mexican scholars to
  the issues which affect both nations; to act as a source of
  information about University-sponsored United States-Mexico
  activities; to develop new sources for support of research and
  instructional programs; and to promote a better understanding
  between the two countries.
\end{enumerate}

\end{document}

%%% Local Variables:
%%% mode: latex
%%% TeX-PDF-mode: t
%%% End:


