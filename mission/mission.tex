\documentclass[12pt]{article}
\usepackage[top=0.8in, bottom=0.8in, right=0.8in, left=0.8in,
paperwidth=8.5in, paperheight=11in, nohead]{geometry}
\geometry{letterpaper}
\usepackage[pdftex]{graphicx}
\usepackage{color}
\usepackage[normalem]{ulem}
\usepackage{amssymb}
\usepackage{amsmath}
\usepackage{epstopdf}
\usepackage{setspace}
\usepackage{mdwlist}
\usepackage{hyperref}
\usepackage{xr}

\title{Ponisio Lab Mission}
\author{Lauren C. Ponisio}

\begin{document}

\maketitle

\section{Mission statement}
\begin{itemize}
\item Research: We seek mechanisms for slowing or preventing the loss
  of biodiversity, specifically native bees. 
\item Teaching: Encourage people to be good decision-makers
and critical thinkers, and cultivate a sense of environmental
consciousness and biophilia (biodiversity appreciation) in others.
\item Mentorship: Recruit \& reatin new people to
  science through familiarizing students with the scientific process
  and the fundalments of ecology. Empower, celebrate and welcome
  members of underrepresented groups.
\end{itemize}

\section{Core values}

\begin{enumerate}
\item Empowerment \& development: a focus on believing that with the
  appropriate dedication and practice we can attain any skills and
  knowledge we desire.
\item Value and promote diversity: a diversity of backgrounds and
  experiences is essential for the progression of science.
\item Collaboration: community and collaboration are key to sustaining
  success in scholarly inquiry. We value and seek collaborations
  inside and outside the lab with relevant stakeholders, including
  government agencies, non-profits and land managers.
\item Collegiality \& humanity: a healthy community is one in which
  collegiality, respect, kindness and empathy are guiding principles.
\item Sharing: data, code, and protocols should be shared whenever possible,
  in ways that enable reuse and discovery.
\item Engagement: communicating science in a variety of formats to
  diverse audiences is an essential component of being an ecologist.
\end{enumerate}


\end{document}

%%% Local Variables:
%%% mode: latex
%%% TeX-PDF-mode: t
%%% End:


